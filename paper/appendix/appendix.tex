\section{Syntax}
The full syntax of \TLLC{} is presented below.

\begin{figure}[H]
  \begin{tabular}{r l c l}
    variables & $x, y, z$   &     &                 \\
    channels  & $c, d$      &     &                 \\
    sorts     & $s, r, t$   & ::= & $\Un$ | $\Ln$   \\
    actions   & $\rho$      & ::= & $!$ | $?$       \\
    terms     & $m,n,A,B,C$ & ::= & $x$ | $c$ | $s$ \\
              &             & \;| & $\PiR{t}{x : A}{B}$ | $\PiI{t}{x : A}{B}$
                                    | $\SigR{t}{x : A}{B}$ | $\SigI{t}{x : A}{B}$ \\
              &             & \;| & $\lamR{t}{x : A}{m}$ | $\lamI{t}{x : A}{m}$
                                    | $\pairR{m}{n}{t}$ | $\pairI{m}{n}{t}$ \\
              &             & \;| & $\appR{m}{n}$ | $\appI{m}{n}$ | $\SigElim{[z]A}{m}{[x,y]n}$ | $\fix{x : A}{B}$ \\
              &             & \;| & $\unit$ | $\ii$ | $\Bool$ | $\bTrue$ | $\bFalse$
                                    | $\boolElim{[z]A}{m}{n_{1}}{n_{2}}$ \\
              &             & \;| & $\CM{A}$ | $\return{m}$ | $\letin{x}{m}{n}$ \\
              &             & \;| & $\Proto$ | $\End$
                                    | $\ActR{\rho}{x : A}{B}$ | $\ActI{\rho}{x : A}{B}$ | $\CH{A}$ | $\HC{A}$ \\
              &             & \;| & $\fork{x : A}{m}$ | $\recvR{m}$ | $\recvI{m}$
                                    | $\sendR{m}$ | $\sendI{m}$ \\
              &             & \;| & $\close{m}$ | $\wait{m}$ \\
    values    & $u, v$      & ::= & $x$ | $c$ | $\lamR{t}{x : A}{m}$ | $\lamI{t}{x : A}{m}$
                                    | $\pairR{u}{v}{t}$ | $\pairI{v}{m}{t}$ \\
              &             & \;| & $\ii$ | $\bTrue$ | $\bFalse$ | $\return{v}$ | $\letin{x}{v}{m}$ \\
              &             & \;| & $\fork{x : A}{m}$ | $\recvR{v}$ | $\recvI{v}$ | $\sendR{v}$ | $\sendI{v}$ \\
              &             & \;| & $\appR{\sendR{v}}{u}$ | $\appI{\sendI{v}}{m}$ | $\close{v}$ | $\wait{v}$ \\
    process   & $O, P, Q$   & \;| & $\langle m \rangle$ | ($P \mid Q$) | $\scope{cd}{P}$
  \end{tabular}
\end{figure}

\section{Auxiliary Operators}
\paragraph{\textbf{Sort Ordering}}
\paragraph{\textbf{Context Merge}}
\paragraph{\textbf{Context Restriction}}

