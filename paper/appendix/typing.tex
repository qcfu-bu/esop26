In this section, we present the full typing rules of \TLLC{}.
We organize the typing rules into logical rules and program rules.

\subsection{Logical Level}
The typing judgment for the logical level has the form $\Gamma \vdash m : A$.
This judgment states that under the \emph{logical context} $\Gamma$, term $m$
has type $A$. 

\paragraph{\textbf{Logical Context}}
The logical context $\Gamma$ is a sequence of variable bindings of the form 
$x_0 : A_0, x_1 : A_1, \dots, x_n : A_n$. Each variable $x_i$ is bound to a type $A_i$.
Variables in the logical context are unrestricted and can be used arbitrarily many
times. The empty context is denoted by $\epsilon$. To ensure the validity of types
in the logical context, we define the context validity judgment $\Gamma \vdash $.
\begin{mathpar}
  \inferrule[Ctx-Empty]
  { }
  { \epsilon \vdash }

  \inferrule[Ctx-Var]
  { \Gamma \vdash \quad \Gamma \vdash A : s \\
    x \notin \Gamma }
  { \Gamma, x : A \vdash }
\end{mathpar}
Note that the context validity judgment is \emph{mutually inductively} defined with the typing judgment. 

\paragraph{\textbf{Core Typing}}