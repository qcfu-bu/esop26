The session fidelity property ensures that well-typed processes 
will adhere to the protocols specified by their session types during execution.
To prove this property, we first must prove that structural congruence preserves typing.

\begin{lemma}[Congruence]\label[lemma]{lemma:congruence}
  Given $\Theta \Vdash P$ and $P \equiv Q$, then $\Theta \Vdash Q$.  
\end{lemma}
\begin{proof}
  By induction on the derivation of $\Theta \Vdash P$ and case analysis on the congruence relation.

\textbf{Case} (\textsc{Expr}): Trivial.

\textbf{Case} (\textsc{Par}):
  \begin{mathpar}
    \inferrule[Par]
    { \Theta_1 \Vdash P \\ \Theta_2 \Vdash Q }
    { \Theta_1 \dotcup \Theta_2 \Vdash P \mid Q }
  \end{mathpar}
  By case analysis on the congruence relation we have the following sub-cases:
  \begin{enumerate}
    \item $P \mid Q \equiv Q \mid P$
    \item $P \mid (Q_1 \mid Q_2) \equiv (P \mid Q_1) \mid Q_2$
    \item $P \mid \proc{\return{\ii}} \equiv P$
    \item $\scope{cd}{P} \mid Q \equiv \scope{cd}{(P \mid Q)}$
  \end{enumerate}

  In sub-case (1), by \textsc{Par} we have $\Theta_2 \dotcup \Theta_1 \Vdash Q \mid P$.
  By the commutativity of $\dotcup$, we have $\Theta_1 \dotcup \Theta_2 \Vdash Q \mid P$ which
  concludes this sub-case.

  In sub-case (2), we have $\Theta_2 \Vdash Q_1 \mid Q_2$. By inversion on its typing derivation, we have
  $\Theta_{21} \Vdash Q_1$ and $\Theta_{22} \Vdash Q_2$ such that $\Theta_2 = \Theta_{21} \dotcup \Theta_{22}$.
  By \textsc{Par} we have $\Theta_1 \dotcup \Theta_{21}  \Vdash P \mid Q_1$.
  By \textsc{Par} again we have $(\Theta_1 \dotcup \Theta_{21}) \dotcup \Theta_{22} \Vdash (P \mid Q_1) \mid Q_2$.
  By the associativity of $\dotcup$, we have $\Theta_1 \dotcup (\Theta_{21} \dotcup \Theta_{22}) \Vdash (P \mid Q_1) \mid Q_2$.
  By substituting $\Theta_2$, we have $\Theta_1 \dotcup \Theta_2 \Vdash (P \mid Q_1) \mid Q_2$ which concludes this sub-case.

  In sub-case (3), we have $Q = \proc{\return{\ii}}$.
  By assumption we have $\Theta_2 \Vdash \proc{\return{\ii}}$.
  By inversion on its typing derivation, we have $\Theta_2 ; \epsilon ; \epsilon \vdash \return{\ii} : \CM{\unit}$.
  By \Cref{lemma:program-inversion-return} we have $\Theta_2 ; \epsilon ; \epsilon \vdash \ii : \unit$.
  By \Cref{lemma:program-inversion-unit} we have $\Theta_2 = \epsilon$ and $\epsilon \triangleright \Un$.
  Thus we have $\Theta_1 \dotcup \Theta_2 \Vdash P$.

  In sub-case (4), we have $\Theta_1 \Vdash \scope{cd}{P}$ and $\Theta_2 \Vdash Q$ by assumption.
  By inversion on the typing derivation of $\Theta_1 \Vdash \scope{cd}{P}$, we have
  $\Theta_1, c \tL \CH{A}, d \tL \HC{A} \Vdash P$ for some protocol $A$.
  By \textsc{Par} we have $\Theta_1, c \tL \CH{A}, d \tL \HC{A} \dotcup \Theta_2 \Vdash P \mid Q$.
  Since $c$ and $d$ are not in $\Theta_2$, we have $(\Theta_1 \dotcup \Theta_2), c \tL \CH{A}, d \tL \HC{A} \Vdash P \mid Q$.
  By \textsc{Scope} we have $\Theta_1 \dotcup \Theta_2 \Vdash \scope{cd}{(P \mid Q)}$ which concludes this sub-case.

\textbf{Case} (\textsc{Scope}):
  \begin{mathpar}
    \inferrule[Scope]
    { \Theta, c \tL \CH{A}, d \tL \HC{A} \Vdash P }
    { \Theta \Vdash \scope{cd}{P} }
  \end{mathpar}

  By case analysis on the congruence relation we have the following sub-cases:
  \begin{enumerate}
    \item $\scope{cd}{(P \mid Q)} \equiv \scope{cd}{P} \mid Q$
    \item $\scope{cd}{P} \equiv \scope{dc}{P}$
    \item $\scope{cd}{\scope{c'd'}{P}} \equiv \scope{c'd'}{\scope{cd}{P}}$
  \end{enumerate}
  
  In sub-case (1), we have $\Theta, c \tL \CH{A}, d \tL \HC{A} \Vdash P \mid Q$ by assumption.
  By inversion on its typing derivation, there exists $\Theta_1$ and $\Theta_2$ such that
  $\Theta = \Theta_1 \dotcup \Theta_2$ and $\Theta_1, c \tL \CH{A}, d \tL \HC{A} \Vdash P$ and $\Theta_2 \Vdash Q$.
  Channels $c$ and $d$ must be distributed the typing judgment $P$ as
  they are linear and do not appear in $Q$. Applying \textsc{Scope} to $\Theta_1, c \tL \CH{A}, d \tL \HC{A} \Vdash P$ we have
  $\Theta_1 \Vdash \scope{cd}{P}$. By \textsc{Par} we have $\Theta_1 \dotcup \Theta_2 \Vdash \scope{cd}{P} \mid Q$
  which concludes this sub-case.

  In sub-case (2), we have $\Theta, c \tL \CH{A}, d \tL \HC{A} \Vdash P$ by assumption.
  By exchange, we have $\Theta, d \tL \HC{A}, c \tL \CH{A} \Vdash P$.
  By \textsc{Scope} we have $\Theta \Vdash \scope{dc}{P}$ which concludes this sub-case.

  In sub-case (3), we have $\Theta, c \tL \CH{A}, d \tL \HC{A}, c' \tL \CH{A'}, d' \tL \HC{A'} \Vdash P$ by assumption.
  By exchange, we have $\Theta, c' \tL \CH{A'}, d' \tL \HC{A'}, c \tL \CH{A}, d \tL \HC{A} \Vdash P$.
  Applying \textsc{Scope} twice we have $\Theta \Vdash \scope{c'd'}{\scope{cd}{P}}$ which concludes this sub-case.
\end{proof}

\begin{theorem}[Session Fidelity]\label{theorem:fidelity}
  If $\Theta \Vdash P$ and $P \Rrightarrow Q$, then $\Theta \Vdash Q$.
\end{theorem}
\begin{proof}
  By induction on the derivation of $P \Rrightarrow Q$ and case analysis on the typing judgment.

\noindent
\textbf{Case} (\textsc{Proc-Fork}):
  \begin{align*}
    \proc{\letin{x}{\fork{y : A}{m}}{n}}
    \Rrightarrow
    \scope{cd}{(\proc{n[c/x]} \mid \proc{m[d/y]})} 
  \end{align*}
  By inversion on $\Theta \Vdash \proc{\letin{x}{\fork{y : A}{m}}{n}}$ we have
  $$\Theta ; \epsilon ; \epsilon \vdash \letin{x}{\fork{y : A}{m}}{n} : \CM{\unit}$$
  By \Cref{lemma:program-inversion-bind} we have
  $\Theta_1 ; x : B ; x :_r B \vdash n : \CM{\unit}$ and
  $\Theta_2 ; \epsilon ; \epsilon \vdash \fork{y : A}{m_0} : \CM{B}$ and $\Theta = \Theta_1 \dotcup \Theta_2$.
  Applying \Cref{lemma:program-inversion-fork} to the later we have
  $${\Theta_2 ; y : \CH{A'} ; y :_t \CH{A'} \vdash m_0 : \CM{\unit}}$$
  and $A = \CH{A'}$ and $\CM{B} \simeq \CM{\HC{A'}}$.
  By the validity of context ${\Theta_2 ; y : \CH{A'} ; y :_t \CH{A'} \vdash}$ we have
  $\epsilon \vdash \CH{A'} : t$. By \textsc{ChType} and \Cref{theorem:sort-uniqueness} we have $t = \Ln$.
  Applying \Cref{corollary:inj-monad} to $\CM{B} \simeq \CM{\HC{A'}}$ we have $B \simeq \HC{A'}$.
  Applying \Cref{corollary:program-context-conv-explicit} to 
  ${\Theta_1 ; x : B ; x :_r B \vdash n : \CM{\unit}}$ using $B \simeq \HC{A'}$ we have
  $${\Theta_1 ; x : \HC{A'} ; x :_r \HC{A'} \vdash n : \CM{\unit}}$$
  By the validity of context ${\Theta_1 ; x : \HC{A'} ; x :_r \HC{A'} \vdash}$ we have
  $\epsilon \vdash \HC{A'} : r$. By \textsc{HcType} and \Cref{theorem:sort-uniqueness} we have $r = \Ln$.

  \noindent
  By \textsc{Channel-HC}, for some fresh channel $c$ we have
  $c \tL \HC{A'} ; \epsilon ; \epsilon \vdash c : \HC{A'}$.

  \noindent
  By \textsc{Channel-HC}, for some fresh channel $d$ we have
  $d \tL \CH{A'} ; \epsilon ; \epsilon \vdash d : \CH{A'}$.

  \noindent
  By \Cref{lemma:program-subst-explicit} we have ${\Theta_1, c \tL \HC{A'} ; \epsilon ; \epsilon  \vdash n[c/x] : \CM{\unit}}$.

  \noindent
  By \Cref{lemma:program-subst-explicit} we have ${\Theta_2, d \tL \CH{A'} ; \epsilon ; \epsilon  \vdash m_0[d/y] : \CM{\unit}}$.

  \noindent
  By \textsc{Expr} we have $\Theta_1, c \tL \HC{A'} \Vdash \proc{n[c/x]}$ and
  $\Theta_2, d \tL \CH{A'} \Vdash \proc{m_0[d/y]}$.

  \noindent
  By \textsc{Par} we have
  ${\Theta_1, c \tL \HC{A'} \dotcup \Theta_2, d \tL \CH{A'} \Vdash \proc{n[c/x]} \mid \proc{m_0[d/y]}}$.

  \noindent
  Since $c$ and $d$ are fresh, we have
  ${(\Theta_1 \dotcup \Theta_2), c \tL \HC{A'}, d \tL \CH{A'} \Vdash \proc{n[c/x]} \mid \proc{m_0[d/y]}}$.

  \noindent
  By \textsc{Scope} we have
  ${\Theta_1 \dotcup \Theta_2 \Vdash \scope{cd}{(\proc{n[c/x]} \mid \proc{m_0[d/y]})}}$
  which concludes this case.

\noindent
\textbf{Case} (\textsc{Proc-End}):
  \begin{align*}
    &\scope{cd}{(\proc{\letin{x}{\close{c}}{m}} \mid \proc{\letin{y}{\wait{d}}{n}})} \\
    &\quad\Rrightarrow 
     \proc{\letin{x}{\return{\ii}}{m}} \mid \proc{\letin{y}{\return{\ii}}{n}} 
  \end{align*}
  By inversion on $\Theta \Vdash \scope{cd}{(\proc{\letin{x}{\close{c}}{m}} \mid \proc{\letin{y}{\wait{d}}{n}})}$ we have either
  \begin{enumerate}
    \item $\Theta, c \tL \CH{A}, d \tL \HC{A} \Vdash \proc{\letin{x}{\close{c}}{m}} \mid \proc{\letin{y}{\wait{d}}{n}}$
    \item $\Theta, c \tL \HC{A}, d \tL \CH{A} \Vdash \proc{\letin{x}{\close{c}}{m}} \mid \proc{\letin{y}{\wait{d}}{n}}$
  \end{enumerate}

  In sub-case (2), by inversion on its typing derivation we have
  ${\Theta_1, c \tL \HC{A} \Vdash \proc{\letin{x}{\close{c}}{m}}}$ and
  ${\Theta_2, d \tL \CH{A} \Vdash \proc{\letin{y}{\wait{d}}{n}}}$ such that ${\Theta = \Theta_1 \dotcup \Theta_2}$.

  \noindent
  By inversion on $\Theta_1, c \tL \HC{A} \Vdash \proc{\letin{x}{\close{c}}{m}}$ we have
  $$\Theta_1, c \tL \HC{A} ; \epsilon ; \epsilon \vdash \letin{x}{\close{c}}{m} : \CM{\unit}$$

  \noindent
  By \Cref{lemma:program-inversion-bind} we have
  $\Theta_{11}, c \tL \HC{A} ; \epsilon ; \epsilon \vdash \close{c} : \CM{B}$.

  \noindent
  By \Cref{lemma:program-inversion-close} we have $\Theta_{11}, c \tL \HC{A} ; \epsilon ; \epsilon \vdash c : \CH{\End}$
  which is a contradiction since $c$ cannot be $\CH{\End}$ in this context. Thus this sub-case is impossible.

  In sub-case (1), by inversion on its typing derivation we have
  ${\Theta_1, c \tL \CH{A} \Vdash \proc{\letin{x}{\close{c}}{m}}}$ and
  ${\Theta_2, d \tL \HC{A} \Vdash \proc{\letin{y}{\wait{d}}{n}}}$ such that ${\Theta = \Theta_1 \dotcup \Theta_2}$.

  \noindent
  By inversion on $\Theta_1, c \tL \CH{A} \Vdash \proc{\letin{x}{\close{c}}{m}}$ we have
  $$\Theta_1, c \tL \CH{A} ; \epsilon ; \epsilon \vdash \letin{x}{\close{c}}{m} : \CM{\unit}$$

  \noindent
  By inversion on $\Theta_2, d \tL \HC{A} \Vdash \proc{\letin{y}{\wait{d}}{n}}$ we have
  $$\Theta_2, d \tL \HC{A} ; \epsilon ; \epsilon \vdash \letin{y}{\wait{d}}{n} : \CM{\unit}$$

  \noindent
  By \Cref{lemma:program-inversion-bind} we have
  $$
    \Theta_{11}, c \tL \CH{A} ; \epsilon ; \epsilon \vdash \close{c} : \CM{B_1}
    \text{ and }
    \Theta_{12} ; x : B_1 ; x :_s B_1 \vdash m : \CM{\unit}
    \text{ and } {\Theta_1 = \Theta_{11} \dotcup \Theta_{12}}
  $$
  By \Cref{lemma:program-inversion-bind} we have
  $$
    \Theta_{21}, d \tL \HC{A} ; \epsilon ; \epsilon \vdash \wait{d} : \CM{B_2}
    \text{ and }
    \Theta_{22}  ; y : B_2 ; y :_t B_2 \vdash n : \CM{\unit}
    \text{ and } {\Theta_2 = \Theta_{21} \dotcup \Theta_{22}}
  $$

  \noindent
  Applying \Cref{lemma:program-inversion-close} on 
  $\Theta_{11}, c \tL \CH{A} ; \epsilon ; \epsilon \vdash \close{c} : \CM{B_1}$ gives us
  $$
    \Theta_{11}, c \tL \CH{A} ; \epsilon ; \epsilon \vdash c : \CH{\End}
    \quad\text{and}\quad
    \CM{B_1} \simeq \CM{\unit}
  $$
  which implies $A \simeq \End$, $B_1 \simeq \unit$, and $\Theta_{11} = \epsilon$.

  \noindent
  Applying \Cref{lemma:program-inversion-wait} on
  $\Theta_{21}, d \tL \HC{A} ; \epsilon ; \epsilon \vdash \wait{d} : \CM{B_2}$ gives us
  $$
    \Theta_{21}, d \tL \HC{A} ; \epsilon ; \epsilon \vdash d : \HC{\End}
    \quad\text{and}\quad
    \CM{B_2} \simeq \CM{\unit}
  $$
  which implies $A \simeq \End$, $B_2 \simeq \unit$ and $\Theta_{21} = \epsilon$.

  \noindent
  By \Cref{corollary:program-context-conv-explicit} we have
  $$
    \Theta_{12} ; x : \unit ; x :_\Un \unit \vdash m : \CM{\unit}
    \quad\text{and}\quad
    \Theta_{22}  ; y : \unit ; y :_\Un \unit \vdash n : \CM{\unit}
  $$

  \noindent
  Since $\Theta_{11} = \epsilon$ and $\Theta_{21} = \epsilon$, we have
  $$
    \Theta_{11} ; \epsilon ; \epsilon \vdash \return{\ii} : \CM{\unit}
    \quad\text{and}\quad
    \Theta_{21} ; \epsilon ; \epsilon \vdash \return{\ii} : \CM{\unit}
  $$

  \noindent
  Applying \textsc{Bind}, we have
  $$\Theta_{11} \dotcup \Theta_{12} ; \epsilon ; \epsilon \vdash \letin{x}{\return{\ii}}{m} : \CM{\unit}$$
  and 
  $$\Theta_{21} \dotcup \Theta_{22} ; \epsilon ; \epsilon \vdash \letin{y}{\return{\ii}}{n} : \CM{\unit}$$

  \noindent
  Applying \textsc{Expr} and \textsc{Par}, we have
  $$(\Theta_{11} \dotcup \Theta_{12}) \dotcup (\Theta_{21} \dotcup \Theta_{22}) \Vdash \proc{\letin{x}{\return{\ii}}{m}} \mid \proc{\letin{y}{\return{\ii}}{n}}$$
  which concludes this case since $(\Theta_{11} \dotcup \Theta_{12}) \dotcup (\Theta_{21} \dotcup \Theta_{22}) = \Theta$.
\end{proof}