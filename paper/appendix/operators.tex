In this section, we define several auxiliary operators/judgments used in the
formalization of \TLLC{}.

\paragraph{\textbf{Sort Ordering}}
The sort ordering relation $\sqsubseteq$ is defined as follows:
\begin{align*}
  (\textsc{Ord-$\Un$})\quad \Un \sqsubseteq s
  \qquad\qquad\qquad
  (\textsc{Ord-$\Ln$})\quad \Ln \sqsubseteq \Ln
\end{align*}
This relation is useful when defining the typing rules of dependent pairs by
ensuring that pairs only contain values of a lower or equal sort.

\paragraph{\textbf{Context Merge}}
The context merge operator $\dotcup$ is a partial function that combines two
contexts into one by selective applying the contraction rule on unrestricted
variables. The operator is undefined if both two contexts contain 
overlapping linear variables.
\begin{mathpar}
  \inferrule[Merge-Empty]
  { }
  { \epsilon \dotcup \epsilon = \epsilon }

  \inferrule[Merge-$\Un$]
  { \Delta_1 \dotcup \Delta_2 = \Delta \\
    x \notin \Delta }
  { (\Delta_1, x \tU A) \dotcup (\Delta_2, x \tU A) = (\Delta, x \tU A) }
  \\

  \inferrule[Merge-$\Ln_1$]
  { \Delta_1 \dotcup \Delta_2 = \Delta \\
    x \notin \Delta }
  { (\Delta_1, x \tL A) \dotcup \Delta_2 = (\Delta, x \tL A) }

  \inferrule[Merge-$\Ln_2$]
  { \Delta_1 \dotcup \Delta_2 = \Delta \\
    x \notin \Delta }
  { \Delta_1 \dotcup (\Delta_2, x \tL A) = (\Delta, x \tL A) }
\end{mathpar}


\paragraph{\textbf{Context Restriction}}
The context restriction operator $\triangleright$ is a predicate that is
useful for defining the typing rules of $\lambda$-expressions. In particular,
it prevents unrestricted functions from capturing linear variables in their closures.
\begin{mathpar}
  \inferrule[ReEmpty]
  { }
  { \epsilon \triangleright s }

  \inferrule[Re-$\Un$]
  { \Delta \triangleright \Un }
  { \Delta, x \tU A \triangleright \Un }

  \inferrule[Re-$\Ln$]
  { \Delta \triangleright \Ln }
  { \Delta, x \ty{s} A \triangleright \Ln }
\end{mathpar}

\paragraph{\textbf{Arity}}
For types $A$ and $X$, we say that $A$ is an \emph{arity} ending on $X$ if it
is either $X$ itself or a $\Pi$-type whose codomain is an arity ending on $X$.
Formally, we define the judgment $A~\arity{X}$ as follows:
\begin{mathpar}
  \inferrule[Arity-Base]
  { }
  { X~\arity{X} }

  \inferrule[Arity-Implicit]
  { B~\arity{X} }
  { \PiI{t}{x : A}{B}~\arity{X} }

  \inferrule[Arity-Explicit]
  { B~\arity{X} }
  { \PiR{t}{x : A}{B}~\arity{X} }
\end{mathpar}
This judgment is used for defining the typing rule of (parameterized) recursive protocols.

\paragraph{\textbf{Guarded}}
For variable $x$ and term $m$, we say that $x$ is \emph{guarded} in $m$ if
the judgment $m~\guard{x}$ is derivable. Intuitively, this means that every 
occurrence of $x$ in $m$ appears under a protocol action. This is important 
for ensuring that recursive protocols do not unfold indefinitely without
performing any actions. The judgment is defined as follows:
\begin{mathpar}\footnotesize
  \inferrule[Guard-Var]
  { x \neq y }
  { y~\guard{x} }

  \inferrule[Guard-Sort]
  { }
  { s~\guard{x} }

  \inferrule[Guard-Implicit-Fun]
  { A~\guard{x} \\ 
    B~\guard{x} }
  { \PiI{t}{x' : A}{B}~\guard{x} }

  \inferrule[Guard-Explicit-Fun]
  { A~\guard{x} \\ 
    B~\guard{x} }
  { \PiR{t}{x' : A}{B}~\guard{x} }

  \inferrule[Guard-Implicit-Lam]
  { A~\guard{x} \\ 
    m~\guard{x} }
  { \lamI{t}{x' : A}{m}~\guard{x} }

  \inferrule[Guard-Explicit-Lam]
  { A~\guard{x} \\ 
    m~\guard{x} }
  { \lamR{t}{x' : A}{m}~\guard{x} }

  \inferrule[Guard-Implicit-App]
  { m~\guard{x} \\ 
    n~\guard{x} }
  { \appI{m}{n}~\guard{x} }

  \inferrule[Guard-Explicit-App]
  { m~\guard{x} \\ 
    n~\guard{x} }
  { \appR{m}{n}~\guard{x} }

  \inferrule[Guard-Implicit-Sum]
  { A~\guard{x} \\ 
    B~\guard{x} }
  { \SigI{t}{x' : A}{B}~\guard{x} }

  \inferrule[Guard-Explicit-Sum]
  { A~\guard{x} \\ 
    B~\guard{x} }
  { \SigR{t}{x' : A}{B}~\guard{x} }

  \inferrule[Guard-Implicit-Pair]
  { m~\guard{x} \\ 
    n~\guard{x} }
  { \pairI{m}{n}{t}~\guard{x} }

  \inferrule[Guard-Explicit-Pair]
  { m~\guard{x} \\ 
    n~\guard{x} }
  { \pairR{m}{n}~\guard{x} }

  \inferrule[Guard-Explicit-SumElim]
  { A~\guard{x} \\ 
    m~\guard{x} \\
    n~\guard{x} }
  { \SigElim{[z]A}{m}{[x',y']n} }

  \inferrule[Guard-RecProto]
  { A~\guard{x} \\ 
    m~\guard{x} }
  { \fix{x' : A}{m}~\guard{x} }

  \inferrule[Guard-Unit]
  { }
  { \unit~\guard{x} }

  \inferrule[Guard-UnitVal]
  { }
  { \ii~\guard{x} }

  \inferrule[Guard-Bool]
  { }
  { \Bool~\guard{x} }

  \inferrule[Guard-True]
  { }
  { \bTrue~\guard{x} }

  \inferrule[Guard-False]
  { }
  { \bFalse~\guard{x} }

  \inferrule[Guard-BoolElim]
  { A~\guard{x} \\
    m~\guard{x} \\ 
    n_1~\guard{x} \\
    n_2~\guard{x} }
  { \boolElim{[z]A}{m}{n_{1}}{n_{2}}~\guard{x} }

  \inferrule[Guard-$\C$Type]
  { A~\guard{x} }
  { \CM{A}~\guard{x} }

  \inferrule[Guard-Return]
  { m~\guard{x} }
  { \return{m}~\guard{x} }

  \inferrule[Guard-Bind]
  { m~\guard{x} \\ 
    n~\guard{x} }
  { \letin{x'}{m}{n}~\guard{x} }

  \inferrule[Guard-Proto]
  { }
  { \Proto~\guard{x} }

  \inferrule[Guard-End]
  { }
  { \End~\guard{x} }

  \inferrule[Guard-Implicit-Action]
  { A~\guard{x} }
  { \ActI{\rho}{x' : A}{B}~\guard{x} }

  \inferrule[Guard-Explicit-Action]
  { A~\guard{x} }
  { \ActR{\rho}{x' : A}{B}~\guard{x} }

  \inferrule[Guard-CH]
  { A~\guard{x} }
  { \CH{A}~\guard{x} }

  \inferrule[Guard-HC]
  { A~\guard{x} }
  { \HC{A}~\guard{x} }

  \inferrule[Guard-Channel]
  { }
  { c~\guard{x} }

  \inferrule[Guard-Fork]
  { A~\guard{x} \\ 
    m~\guard{x} }
  { \fork{x': A}{m}~\guard{x} }

  \inferrule[Guard-Implicit-Recv]
  { m~\guard{x} }
  { \recvI{m}~\guard{x} }

  \inferrule[Guard-Explicit-Recv]
  { m~\guard{x} }
  { \recvR{m}~\guard{x} }

  \inferrule[Guard-Implicit-Send]
  { m~\guard{x} }
  { \sendI{m}~\guard{x} }

  \inferrule[Guard-Explicit-Send]
  { m~\guard{x} }
  { \sendR{m}~\guard{x} }

  \inferrule[Guard-Wait]
  { m~\guard{x} }
  { \wait{m}~\guard{x} }

  \inferrule[Guard-Close]
  { m~\guard{x} }
  { \close{m}~\guard{x} }
\end{mathpar}
\clearpage