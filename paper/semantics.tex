\subsection{Process Configurations}
In the previous section, we have presented the typing rules for \TLLC{} terms
which form individual processes. To compose multiple processes together, we
introduce the process level typing judgment $\Theta \Vdash P$ below. This judgment
formally states that a configuration of processes $P$ is well-typed under the
context $\Theta$, which tracks the channels used by the processes in $P$.
\begin{mathpar}
  \inferrule[Expr] 
  { \Theta ; \epsilon ; \epsilon \vdash m : \CM{\unit} }
  { \Theta \Vdash \proc{m} }

  \inferrule[Par]
  { \Theta_1 \Vdash P_1 \\ 
    \Theta_2 \Vdash P_2 }
  { \Theta_1 \dotcup \Theta_2 \Vdash P_1 \mid P_2 }

  \inferrule[Scope]
  { \Theta, c :_\Ln \CH{A}, d :_\Ln \HC{A} \Vdash P }
  { \Theta \Vdash \scope{cd}{P} }
\end{mathpar}

The process configuration rules are standard. The \textsc{Expr} rule lifts
well-typed closed terms of type $\CM{\unit}$ to processes. It is important for
the term $m$ to be closed as processes in a configuration cannot rely on
external substitutions to resolve free variables, they can only communicate
through channels. In the \textsc{Par} rule, well-typed configurations $P$ and
$Q$ can be composed in parallel as long as their contexts $\Theta_1$ and
$\Theta_2$ can be combined. The \textsc{Scope} rule allows two dual channels
to be connected together, allowing processes holding channels $c$ and $d$ to communicate.

\subsection{Semantics}
\subsection{Meta-Theory}
\paragraph{\textbf{Session Fidelity}}
\paragraph{\textbf{Erasure Safety}}