\subsection{Process Configurations}
In the previous section, we have presented the typing rules for \TLLC{} terms
which form individual processes. To compose multiple processes together, we
introduce the process level typing judgment $\Theta \Vdash P$ below. This judgment
formally states that a configuration of processes $P$ is well-typed under the
context $\Theta$, which tracks the channels used by the processes in $P$ at runtime.
\begin{mathpar}
  \inferrule[Expr] 
  { \Theta ; \epsilon ; \epsilon \vdash m : \CM{\unit} }
  { \Theta \Vdash \proc{m} }

  \inferrule[Par]
  { \Theta_1 \Vdash P_1 \\ 
    \Theta_2 \Vdash P_2 }
  { \Theta_1 \dotcup \Theta_2 \Vdash P_1 \mid P_2 }

  \inferrule[Scope]
  { \Theta, c :_\Ln \CH{A}, d :_\Ln \HC{A} \Vdash P }
  { \Theta \Vdash \scope{cd}{P} }
\end{mathpar}

The process configuration rules are standard. The \textsc{Expr} rule lifts
well-typed closed terms of type $\CM{\unit}$ to processes. It is important for
the term $m$ to be closed as processes in a configuration cannot rely on
external substitutions to resolve free variables. They can only communicate
through channels. In the \textsc{Par} rule, well-typed configurations $P$ and
$Q$ can be composed in parallel as long as their contexts $\Theta_1$ and
$\Theta_2$ can be combined. The \textsc{Scope} rule allows two dual channels
to be connected together, allowing processes holding channels $c$ and $d$ to communicate.

The structural congruence of process configurations is defined as the least
congruence relation generated by the following standard rules:
\begin{mathpar}
  P \mid Q \equiv Q \mid P 

  O \mid (P \mid Q) \equiv (O \mid P) \mid Q

  P \mid \proc{\return{\ii}} \equiv P
  \\
  \scope{cd}{P} \mid Q \equiv \scope{cd}{(P \mid Q)}

  \scope{cd}{P} \equiv \scope{dc}{P}

  \scope{cd}{\scope{c'd'}{P}} \equiv \scope{c'd'}{\scope{cd}{P}}
\end{mathpar}
Structural congruence states that parallel composition is commutative and
associative and compatible with channel scoping. Processes which terminate
with the unit value $\ii$ can be removed from a configuration.
Intuitively, two structurally congruent configurations should be
considered equivalent regarding their communication behavior.

\subsection{Semantics}
\paragraph{\textbf{Term Reduction}}
The operational semantics of \TLLC{} programs is mostly the same as that of
call-by-value TLL~\cite{fu25}.  The relation $m \Leadsto m'$ is used to denote a
single step of \emph{program} level reduction. Due to the monadic formulation of
concurrency in \TLLC{}, the only additional (non-trivial) program reduction rule
is the following \textsc{BindElim} rule which reduces a monadic
\bsf{let}-expression when its bound term is a \bsf{return} expression:
\begin{align*}
 (\textsc{BindElim})\qquad \letin{x}{\return{v}}{m} \Leadsto m[v/x] \tag*{(\text{where $v$ is a value})}
\end{align*}
Values now additionally include channels, partially applied communication operators
and thunked monadic expressions. We will use the metavariable $v$ to denote values
for the rest of this paper. The full definition of values is presented in the appendix.

\paragraph{\textbf{Process Reduction}}
The semantics of processes is defined through the relation $P \Rrightarrow Q$ which
states that process configuration $P$ reduces to process configuration $Q$
in one step. The process reduction rules are presented below.

\vspace{0.5em}
\begin{small}
\begin{tabular}{l L}
  (\textsc{Proc-Fork}) &
    \proc{\letin{x}{\fork{y : A}{m}}{n}}
    \Rrightarrow
    \scope{cd}{(\proc{n[c/x]} \mid \proc{m[d/y]})} 
  \\
  (\textsc{Proc-End}) 
    &\scope{cd}{(\proc{\letin{x}{\close{c}}{m}} \mid \proc{\letin{y}{\wait{d}}{n}})} \\
    &\quad\Rrightarrow 
     \proc{\letin{x}{\return{\ii}}{m}} \mid \proc{\letin{y}{\return{\ii}}{n}} 
  \\
  (\textsc{Proc-Com}) 
    &\scope{cd}{(\proc{\letin{x}{\appR{\sendR{c}}{v}}{m}} \mid \proc{\letin{y}{\recvR{d}}{n}})} \\
    &\quad\Rrightarrow 
     \scope{cd}{(\proc{\letin{x}{\return{c}}{m}} \mid \proc{\letin{y}{\return{\pairR{v}{d}{\Ln}}}{n}})}
  \\
  (\textsc{Proc-\underline{Com}}) 
    &\scope{cd}{(\proc{\letin{x}{\appI{\sendI{c}}{o}}{m}} \mid \proc{\letin{y}{\recvI{d}}{n}})} \\
    &\quad\Rrightarrow 
     \scope{cd}{(\proc{\letin{x}{\return{c}}{m}} \mid \proc{\letin{y}{\return{\pairI{o}{d}{\Ln}}}{n}})}
\end{tabular}
\vspace{0.2em}
\begin{mathpar}
  \inferrule[(Proc-Expr)]
  { m \Leadsto m' }
  { \proc{m} \Rrightarrow \proc{m'} }

  \inferrule[(Proc-Par)]
  { P \Rrightarrow Q }
  { O \mid P \Rrightarrow O \mid Q }

  \inferrule[(Proc-Scope)]
  { P \Rrightarrow Q }
  { \scope{cd}{P} \Rrightarrow \scope{cd}{Q} }

  \inferrule[(Proc-Congr)]
  { P \equiv P' \\ 
    P' \Rrightarrow Q' \\ 
    Q' \equiv Q }
  { P \Rrightarrow Q }
\end{mathpar}
\end{small}

\noindent
The first four rules define the synchronous communication semantics of \TLLC{}. The
\textsc{Proc-Fork} rule creates a pair of dual channels $c$ and $d$ to connect
the continuation $n$ of the parent process with the newly forked child process $m$.
We can see here that the newly created channels $c$ and $d$ are substituted for 
the variables $x$ and $y$ in $n$ and $m$ respectively.

The \textsc{Proc-End} rule synchronizes the termination of communicating on dual 
channels $c$ and $d$. The resulting process configuration contains two processes
which are no longer connected by any channels. Additionally, the close and wait 
operations are replaced by unit return values once the termination is synchronized.

The \textsc{Proc-Com} rule governs the communication of a real message $v$ from
a sender to a receiver. The sending process continues as $m$ with the channel
$c$ while the receiving process continues as $n$ with the received
message $v$ and the channel $d$ paired together as $\pairR{v}{d}{\Ln}$.

The \textsc{Proc-\underline{Com}} rule is similar to \textsc{Proc-Com} except
that it handles the communication of a ghost message $o$. While this rule seems
to indicate that ghost messages are communicated at runtime, we will later show
through the erasure safety theorem that ghost messages are always safe to be erased.
The exchange of ghost messages here is only for the purpose of establishing a 
reference point for reasoning about the correctness of erasure safety.

The remaining four rules are standard. The \textsc{Proc-Expr} rule allows a
singleton process to reduce by reducing its underlying term. The \textsc{Proc-Par} and
\textsc{Proc-Scope} rules allow a process to reduce in parallel composition and
under channel scope respectively. Finally, the \textsc{Proc-Congr} rule allows
processes to reduce up to structural congruence.

\subsection{Meta-Theory}
\paragraph{\textbf{Compatibility}}
We first show that the concurrency extensions of \TLLC{} are compatible with the
underlying TLL type system. To this end, we prove that \TLLC{} enjoys the same
meta-theoretical properties as TLL. Due to the fact that these properties do not involve 
concurrency, their proofs indicate that \TLLC{} is sound as a term calculus.
Here we present a few representative theorems. The full list of theorems and their proofs
can be found in our appendix.

The first theorem we present is the validity theorem which states that
well-typed terms have well-sorted types. This theorem is important as it
ensures that the types appearing in typing judgments are indeed valid (i.e. they inhabit a sort).
\begin{theorem}[Validity]
  Given $\Theta ; \Gamma ; \Delta \vdash m : A$, there exists
  sort $s$ such that $\Gamma \vdash A : s$.
\end{theorem}

In TLL and \TLLC{}, the sort of a type determines whether the type is a unrestricted or linear.
This means that it is crucial for a type to have a unique sort, otherwise the same type
could be interpreted as both unrestricted and linear, leading to unsoundness. To address this
concern, we prove the sort uniqueness theorem below which states that a type can have at most one sort.
This ensures no ambiguity on whether a type is to be considered unrestricted or linear.
\begin{theorem}[Sort Uniqueness]
  Given $\Gamma \vdash A : s$ and $\Gamma \vdash A : t$,
  we have $s = t$.
\end{theorem}

The next theorem we present is the standard subject reduction theorem which states that
types are preserved under term reduction. This theorem is necessary for ensuring that
session fidelity holds during process reduction as singleton processes reduce by reducing their
underlying terms.
\begin{theorem}[Subject Reduction]
  Given $\Theta ; \epsilon ; \epsilon \vdash m : A$ and
  $m \Leadsto m'$, we have
  $\Theta ; \epsilon ; \epsilon \vdash m' : A$.
\end{theorem}

\paragraph{\textbf{Session Fidelity}}
The session fidelity theorem ensures that processes adhere to the communication
protocols specified by their types. This property guarantees that well-typed
processes will not encounter communication mismatches at runtime. Since we consider
processes up to structural congruence, we must first show that configuration
typing is preserved under structural congruence. This manifests as the following lemma.

\begin{lemma}[Congruence]
  Given $\Theta \Vdash P$ and $P \equiv Q$, we have $\Theta \Vdash Q$.
\end{lemma}

\noindent
The session fidelity theorem is then stated as follows.
\begin{theorem}[Session Fidelity]
  Given $\Theta \Vdash P$ and $P \Rrightarrow Q$, we have $\Theta \Vdash Q$.
\end{theorem}
\noindent

One of the primary challenges in proving session fidelity is to show that typing
is preserved during communication steps, specifically the \textsc{Proc-Com}, and
\textsc{Proc-\underline{Com}} cases. In these cases, the message being
communicated is transported from the sender to the receiver without the use of a
substitution. We need to show that the message, after communication, is
consistently typed with regards to the receiver's context. Unlike simple type
systems where one could simply place a value into any context so long as
the value has the expected type, dependent type systems require more care. For
instance, the evaluation context $\pairR{[\cdot]}{\Refl}{} : \Sigma(x : \textsf{nat}).(x = 1)$ 
is well-typed if and only if the hole is filled with $1$. To address this challenge, 
we design the monadic \textsc{Bind} rule (\Cref{sec:dependent-session-types})
to disallow dependency on the bound value. More specifically, for
$\letin{x}{m}{n}$ expressions, the type of $n$ \emph{cannot} depend on $x$. This
restriction means that $m$ can be replaced by any other expression of the same
type without affecting the type of $n$. Consider the \textsc{Proc-Com} step below:
\begin{align*}
  &\scope{cd}{(\proc{\letin{x}{\appR{\sendR{c}}{v}}{m}} \mid \proc{\letin{y}{\recvR{d}}{n}})} \\
  &\quad\Rrightarrow \scope{cd}{(\proc{\letin{x}{\return{c}}{m}} \mid \proc{\letin{y}{\return{\pairR{v}{d}{\Ln}}}{n}})}
\end{align*}
This operation is carried out between two singleton processes that are
evaluating monadic \bsf{let}-expressions. Due to the dependency restriction of
the \textsc{Bind} rule, we can replace $\appR{\sendR{c}}{v}$ with $\return{c}$
and $\recvR{d}$ with $\return{\pairR{v}{d}{\Ln}}$ without affecting the types of
$m$ and $n$. Due to the fact that all communication operations in \TLLC{} are
carried out on \bsf{let}-expressions, the dependency restriction ensures that
session fidelity holds during communication steps.

\paragraph{\textbf{Global Progress}}
Global progress, i.e. deadlock-freedom, is a desirable property for concurrent programs.
Many session type systems~\cite{wadler12,caires10,das20} guarantee global progress by 
construction through a disciplined use of channels. However, there are also session 
type systems~\cite{honda93,honda16,ldst,balzer17} that eschew global progress in favor of
more expressive session types. \TLLC{} belongs to the latter category if we consider
arbitrary well-typed process configurations. This is because the process type system of
\TLLC{} does not prevent cyclic channel topologies that can lead to deadlocks.
However, we can still prove a weaker form of global progress for \TLLC{} by
considering only \emph{reachable} process configurations. 
Intuitively, reachable configurations are those that can be constructed
by \bsf{fork} operations starting from a single process. The global progress theorem
is then stated as follows.
\begin{theorem}[Global Progress]
  Given $\epsilon \Vdash P$ where $P$ is reachable, either
  \begin{itemize}
    \item $P \equiv \proc{\return{\ii}}$, or
    \item there exists $Q$ such that $P \Rrightarrow Q$.
  \end{itemize}
\end{theorem}

\paragraph{\textbf{Erasure Safety}}
To show that ghost messages are safe to erase, we define an erasure relation
$\Theta ; \Gamma ; \Delta \vdash m \sim m' : A$. This relation states that all
ghost arguments and type annotations in $m$ are replaced by a special opaque 
value $\square$ in $m'$. This relation is similar to the one defined for
the erasure of \emph{propositions} in standard dependent type 
theories~\cite{barras08,letouzey03,sozeau20}.
The most important erasure rule is shown below. The full set of erasure
rules can be found in the appendix.
\begin{mathpar}
  \inferrule[Erase-Implicit-App]
  { \Theta ; \Gamma ; \Delta \vdash m \sim m' : \PiI{t}{x : A}{B} \\ 
    \Gamma \vdash n : A }
  { \Theta ; \Gamma ; \Delta \vdash \appI{m}{n} \sim \appI{m'}{\square} : B[n/x] }
\end{mathpar}

The \textsc{Erase-Implicit-App} rule states that when erasing an implicit
application $\appI{m}{n}$, the ghost argument $n$ is replaced by $\square$ in the
erased term. Consider the $\SendI{}\ c$ operator for sending ghost messages on channel 
$c$. As defined in \Cref{sec:dependent-session-types}, this partially applied operator 
has a type of the form $\PiI{\Ln}{x : A}{\CM{B}}$. When fully applied as $\appI{\SendI{}\ c}{n}$, 
the ghost argument $n$ is erased to $\square$ by \textsc{Erase-Implicit-App}.
Since $\square$ is an opaque value, it cannot be inspected or pattern matched on.
Thus, if programs can be evaluated soundly after erasing all ghost arguments
and type annotations, we can conclude that ghost messages are safe to erase.

The erasure relation is then naturally lifted to the process level as $\Theta \Vdash P \sim P'$
where $P'$ is the erased version of $P$. The rules for this relation are as follows:

\begin{small}
\begin{mathpar}
  \inferrule[Erase-Expr]
  { \Theta ; \epsilon ; \epsilon \vdash m \sim m' : \CM{\unit} }
  { \Theta \Vdash \proc{m} \sim \proc{m'} }

  \inferrule[Erase-Par]
  { \Theta_1 \Vdash P \sim P' \\ 
    \Theta_2 \Vdash Q \sim Q' }
  { \Theta_1 \dotcup \Theta_2 \Vdash (P \mid Q) \sim (P' \mid Q') }

  \inferrule[Erase-Scope]
  { \Theta, c :_\Ln \CH{A}, d :_\Ln \HC{A} \Vdash P \sim P' }
  { \Theta \Vdash \scope{cd}{P} \sim \scope{cd}{P'} }
\end{mathpar}
\end{small}

We show that erasure is safe through the following two theorems. These theorems
tell us that any possible reduction on an original object (either a term or process) 
can be simulated on its erased counterpart. Moreover, the erased object obtained after 
reduction also satisfies the erasure relation with respect to the reduced original object.
Basically, these theorems state that any possible evaluation path of the original object
remains valid after erasure.

\begin{theorem}[Term Simulation]
  Given $\Theta ; \epsilon ; \epsilon \vdash m \sim m' : A$ and $m \Leadsto n$, there exists $n'$ such that
  $m' \Leadsto^* n'$ and $\Theta ; \epsilon ; \epsilon \vdash n \sim n' : A$.
\end{theorem}

\begin{theorem}[Process Simulation]
  Given $\Theta \Vdash P \sim P'$ and reduction $P \Rrightarrow Q$, there exists $Q'$ such that
  $P' \Rrightarrow Q'$ and $\Theta \Vdash Q \sim Q'$.
\end{theorem}