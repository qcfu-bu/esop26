Earlier in the introduction section, we showed a sketch of how dependent session types
can be used for certified concurrent programming through the example of a concurrent queue.
In this section, we provide a detailed account of how we can use dependent session types
to construct a generic map-reduce system. Similarly to the queue example, we will verify
the correctness of the map-reduce system by relating it to sequential operations on trees.

The first step in constructing the map-reduce system is to define the kinds of operations
that can be performed by the system.
\begingroup
\small
\addtolength{\jot}{-0.2em}
\begin{align*}
  \Inductive\ \textsf{opr}(A : \Un) :=\ &\textsf{Free}  : \textsf{opr}(A) \\
  \mid\ &\textsf{Map}   : \forall \{B : \Un\}\ (f : A \rightarrow B) \rightarrow \textsf{opr}(A) \\
  \mid\ &\textsf{Reduce}: \forall \{B : \Un\}\ (f : A \rightarrow B)\ (g : B \rightarrow B \rightarrow B) \rightarrow \textsf{opr}(A)
\end{align*}
\endgroup

%%% Local Variables:
%%% mode: LaTeX
%%% TeX-master: "main"
%%% End:
