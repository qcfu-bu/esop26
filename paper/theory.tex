\subsection{Core TLL}
In this section, we give a brief summary of the Two-Level Linear dependent type theory (TLL)~\cite{fu23}. 
TLL is a dependent type theory that combines 
Martin-L\"{o}f-style dependent types~\cite{martinlof} 
with linear types~\cite{girard,wadler1990}. 
Notably, TLL supports \emph{essential linearity}~\cite{luo} through the use of
a stratified ``two-level'' typing system: the \emph{logical} level and the \emph{program} level. 
The typing judgments of the two levels are written as follows:
\begin{center}
\vspace{0.5em}
\begin{tikzpicture}[
    node distance=2.4cm,
    >=stealth, auto,
    every state/.style={rectangle, draw, rounded corners}
]
\node[state, fill=blue!5] (l)                {\small$\Gamma \vdash m : A\ \text{(Logical Typing)}$};
\node[state, fill=red!5]  (p) [right=of l]   {\small$\Gamma ; \Delta \vdash m : A\ \text{(Program Typing)}$};
\path[-latex,transform canvas={yshift=+1.5ex}] (l.east) edge node {\footnotesize{provides types}} (p.west);
\path[-latex,transform canvas={yshift=-1.5ex}] (p.west) edge node {\footnotesize{subjects to verify}} (l.east);
\end{tikzpicture}
\vspace{0.5em}
\end{center}

First, the \emph{logical} level is a standard dependent type system that supports unrestricted 
usage of types and terms. The primary purpose of the logical level is to provide typing rules
for types which will be used at the logical level. For example, the rules for dependent 
function type ($\Pi$-types) formation are defined at the logical level as follows:
\begin{mathpar}
  \inferrule[Explicit-Fun]
  { \Gamma \vdash A : s \\
    \Gamma, x : A \vdash B : r }
  { \Gamma \vdash \PiR{t}{x : A}{B} : t }

  \inferrule[Implicit-Fun]
  { \Gamma \vdash A : s \\
    \Gamma, x : A \vdash B : r }
  { \Gamma \vdash \PiI{t}{x : A}{B} : t }
\end{mathpar}
The symbols $s, r, t$ range over the \emph{sorts} of type universes, i.e. 
$\Un$ or $\Ln$. These sorts are used to classify types into two categories: 
unrestricted types ($A : \Un$) and linear types ($A : \Ln$).
Program level terms which inhabit unrestricted types can be freely duplicated or discarded,
while those which inhabit linear types must be used exactly once.
Note that this usage restriction is \emph{not} enforced at the logical level
as the logical level typing judgment is completely structural.
This is safe because the logical level will never be executed at runtime and 
is only used for type checking and verification. Thus, multiple uses of
a linear resource at the logical level will not lead to any runtime errors.

At the program level, the typing judgment $\Gamma ; \Delta \vdash m : A$ is used to
exclusively type \emph{terms}. In other words, no rules for forming types are defined
at the program level. All the types used in $\Gamma$, $\Delta$, $m$ and $A$ must be well-formed
according to the logical level typing judgment. This typing judgment possesses two contexts:
$\Gamma$ of all variables in scope, and $\Delta$ of all variables that are computationally relevant
in program $m$. Context $\Delta$ is crucial for enforcing linearity at the program level.
For example, consider the $\lambda$-abstraction rules:
\begin{mathpar}
  \inferrule[Explicit-Lam]
  { \Gamma, x : A ; \Delta, x :_s A \vdash m : B \\ 
    \Delta \triangleright t }
  { \Gamma ; \Delta \vdash \lamR{t}{x : A}{m} : \PiR{t}{x : A}{B} }

  \inferrule[Implicit-Lam]
  { \Gamma, x : A ; \Delta \vdash m : B \\
    \Delta \triangleright t }
  { \Gamma ; \Delta \vdash \lamI{t}{x : A}{m} : \PiI{t}{x : A}{B} }
\end{mathpar}
In \textsc{Explicit-Lam}, we can see that the bound variable $x$ is added to
both contexts $\Gamma$ and $\Delta$. This indicates that $x$ is a variable which
can be used both logically (in types and ghost values) through $\Gamma$, and
computationally (in real values) through $\Delta$. On the other hand, in the
\textsc{Implicit-Lam} rule, $x$ is only added to $\Gamma$ but not $\Delta$.
This indicates that $x$ is a ghost variable which can only be used logically.
A ubiquitous example of ghost variables are type parameters in polymorphic functions.
For example, the polymorphic identity function can be implemented as
\begin{align*}
  \lamI{\Un}{A : \Un}{\lamR{\Un}{x : A}{x}}
\end{align*}
which has the type $\PiI{\Un}{A : \Un}{\PiR{\Un}{x : A}{A}}$.
Arguments to implicit functions are typed at the logical level, thus
allowing polymorphic functions to be instantiated with a type as an argument.
Additionally, as demonstrated in the examples of prior sections,
ghost variables also facilitate program verification by statically describing 
abstractions and invariants of program states.

In the two $\lambda$-abstraction rules above, 
the premise $\Delta \triangleright t$ is a simple side condition that states: if
$t = \Un$, then all variables in $\Delta$ must be unrestricted. In other words,
the $\lambda$-abstractions that can be applied unrestrictedly (with $t = \Un$)
are not allowed to capture linearly typed variables from $\Delta$. This is
similar to the restriction imposed on closures implementing the $\textsf{Fn}$
trait (i.e. those that can be called multiple times) in Rust~\cite{rust} where
capturing of mutable references is prohibited. If such a restriction is not
imposed, then evaluating a $\lambda$-abstraction (that captures a linear
variable) twice may lead to unsafe memory accesses such as double frees or
use-after-frees.

The application rules for both explicit and implicit functions are as follows:
\begin{mathpar}
  \inferrule[Explicit-App]
  { \Gamma ; \Delta_1 \vdash m : \PiR{t}{x : A}{B} \\ 
    \Gamma ; \Delta_2 \vdash n : A }
  { \Gamma ; \Delta_1 \dotcup \Delta_2 \vdash \appR{m}{n} : B[n/x] }

  \inferrule[Implicit-App]
  { \Gamma ; \Delta \vdash m : \PiI{t}{x : A}{B} \\ 
    \Gamma \vdash n : A }
  { \Gamma ; \Delta \vdash \appI{m}{n} : B[n/x] }
\end{mathpar}
In \textsc{Explicit-App}, the argument $n$ is a real value which must be typed
at the program level. The $\dotcup$ operator merges the two program context
$\Delta_1$ and $\Delta_2$ by contracting unrestricted variables and requiring
that linear variables be disjoint, thus preventing the sharing of linear
resources. In \textsc{Implicit-App}, the argument $n$ is a ghost value that is
typed at the logical level. Due to the fact that ghost values are erased prior
to runtime, the program context $\Delta$ in the conclusion only tracks the
computationally relevant variables used in $m$. Notice how in \textsc{Explicit-App}, 
the argument $n$ is substituted into the return type $B$. This allows types to depend 
on program level terms regardless of whether they are of linear or unrestricted types.

\paragraph{\textbf{Usage vs Uniqueness}}
Compared to other linear dependent type
theories~\cite{qtt,nothing,llf,vakar14,luo} which only enforce the linear
\emph{usage} of resources, the TLL type system prevents the \emph{sharing} of
linear resources as well. This is similar to the subtle difference between
linear logic~\cite{girard} and bunched implications~\cite{ohearn99,ohearn03}. 
Consider a linear function $f$, in the aforementioned dependent type theories,
of some type $A \multimap B$. When function $f$ is applied to some argument $v$
of type $A$, the argument $v$ is guaranteed to be used exactly once in the 
\emph{body} of $f$. Notice that this notion of linearity does not preclude the
possibility of sharing $v$ with other functions or threads outside of $f$. In
contrast, the shared interpretation of linearity, as described by
O'Hearn~\cite{ohearn03}, require that $v$ be a unique resource which is
exclusively owned by $f$. In other words, $v$ cannot be shared with any other
functions or threads outside of $f$. Wadler, in his seminal
work~\cite{wadler1991}, also made a similar distinction between \emph{linearity}
and \emph{uniqueness} in the context of functional programming, coining the term
\emph{steadfast types} to refer to type systems that enforce both linearity and
uniqueness. In the context of concurrency, the steadfast type system of TLL
makes it especially suitable for integration with session types: linear usage
prevents replaying of communication protocols and uniqueness prevents multiple
threads from simultaneously accessing the same communication channel.

\subsection{Dependent Session Types of \TLLC{}}
In this section, we formally present the dependent session types of \TLLC{}.
\paragraph{\textbf{Protocols and Channel Types}}
\paragraph{\textbf{Concurrency Monad}}